\begin{center}
{\normalsize
Министерство образования и науки Российской Федерации\\
Федеральное государственное бюджетное образовательное\\
учреждение высшего профессионального образования\\
 «Самарский государственный аэрокосмический университет\\
 имени академика С.П. Королева\\
 (национальный исследовательский университет)\\[10pt]
Факультет информатики \\
Кафедра технической кибернетики
}
\end{center}

\begin{flushright}
\begin{tabular}{c}
<<УТВЕРЖДАЮ>>\\
Заведующий кафедрой\\
\underline{\phantom{заведующий}} Фамилия И.О.\\
<<\underline{\phantom{301}}>>\underline{\phantom{февраль}} 2014 г.
\end{tabular}
\end{flushright}

\begin{center}
{\bfseries
ЗАДАНИЕ\\
по подготовке магистерской диссертации\\
}
студента группы XXXX Фамилия Имя Отчество
\end{center}
\noindent1. Тема работы: <<Реализация алгоритма YYYY в системе YYYY>> утверждена приказом по университету от «14» марта 2014 г. №95-ст.

\noindent2. Исходные данные к работе: книга Григорий Остер <<Вредные советы>>, диссертация Ilgner R. G. <<A comparative analysis of the performance and deployment overhead of parallelized Finite Difference Time Domain (FDTD) algorithms on a selection of high performance multiprocessor computing systems>>, пакет Meep.

\noindent3. Перечень вопросов, подлежащих разработке в работе: 
    \begin{enumerate}
        \item Реализация последовательного алгоритма YYYY для YYYY случая на языке C.  
        \item Разработка и реализация параллельных алгоритмов на основе технологии YYYY с одномерным и двумерным разбиением сеточной области по пространству.
        \item Исследование эффективности предложенных реализаций методом вычислительного эксперимента.
    \end{enumerate}
\pagebreak
\noindent4. График выполнения работы:\\
        \begin{tabular}{|p{4.7cm}|p{1cm}|p{2.5cm}|p{2cm}|p{2cm}|p{2cm}|}  \hline
        \multirow{2}{4cm}{Этапы работы}	& \multirow{2}{*}{\%} &\multirow{2}{2.5cm}{Сроки выполнения по этапам}	&\multicolumn{3}{|c|}{Итоги проверки}\\ \cline{4-6}
        &&& Отметка о вып.	&Подпись магист-та	&Подпись рук-ля\\ \hline
         Реализация последовательного алгоритма YYYY  &10	&28.02.2014	&вып.&&	\\	 \hline
        Разработка параллельного алгоритма с YYYY декомпозицией &20	&10.03.2014	&вып.&&   \\   \hline
        Разработка параллельного алгоритма с YYYY декомпозицией         &30	&20.03.2014	&вып.&&	\\	 \hline
        Разработка конвейрного алгоритма &50	&10.04.2014	&вып.&&	\\	 \hline
        Программная реализация и исследование эффективности разработанных алгоритмов  &70	&25.04.2014	&вып.&&	\\	 \hline
        Исследование эффективности разработанных алгоритмов в сравнении с пакетом Meep 	 &90	&15.05.2014	&вып.&&	\\	 \hline
        Подготовка документации по магистерской диссертации	&100	&25.05.2014	&вып. &&\\ \hline
        \end{tabular}\\[10pt]

\noindent5.  Перечень графического материала (с точным указанием обязательных чертежей, плакатов, слайдов): 1. Титульный слайд. 2. Актуальность. 3. Уравнения Максвелла. 4. Алгоритм YYYY. 5. Метод построения параллельных программ. 6. Параллельная версия с YYYY декомпозицией. 7. Результаты выполнения алгоритма с YYYY декомпозицией 8. Параллельная версия с YYYY декомпозицией. 9. Результаты выполнения алгоритма с YYYY декомпозицией 10. YYYY версия. 11. Результаты выполнения YYYY алгоритма 12. Выводы\\[10pt]


Срок представления законченной работы: <<\underline{\phantom{232}}>> мая 2014 г.\\[10pt]

Дата выдачи задания: <<\underline{\phantom{232}}>> \underline{\phantom{февраль}} 20\underline{\phantom{23}} г.\\[10pt]

Руководитель работы  \underline{\phantom{длинная подпись}}  Фамилия И.О.\\[10pt]

Задание принял к исполнению <<\underline{\phantom{232}}>> \underline{\phantom{февраль}} 20\underline{\phantom{23}} г.\\[5pt]
\phantom{Задание принял к исполнению}\underline{\phantom{длинная подпись}}

\newpage
